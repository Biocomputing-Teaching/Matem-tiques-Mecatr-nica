\Exercise Determinar si els vectors $\uvec{u}$ i $\uvec{v}$ són paral·lels, ortogonals o cap de les dues coses:

\begin{enumerate}
  \item $\uvec{u}=(1,-2)$ i $\uvec{v}=(2.-4)$
  \item $\uvec{u}=(1,-2,1)$ i $\uvec{v}=(2,1,0)$
  \item $\uvec{u}=\uvec{j}+6\uvec{k}$, $\uvec{v}=\uvec{i}-2\uvec{j}-\uvec{k}$
\end{enumerate}

\Answer Per determinar el paral·lelisme, observem si un dels dos vectors és proporcional a l'altre. Per determinar l'ortogonalitat en fem el producte escalar:

\begin{enumerate}
  \item $\uvec{u}=(1,-2)$ i $\uvec{v}=(2,-4)$. Veiem que $\uvec{v}=2\uvec{u}$ i, per tant, són paral·lels.
  \item $\uvec{u}=(1,-2,1)$ i $\uvec{v}=(2,1,0)$. No són paral·lels perquè no són proporcionals. Per saber si són ortogonals en calculem el producte escalar: $\uvec{u}\cdot\uvec{v}=1\cdot2+(-2)\cdot1+1\cdot0=0$. Per tant, són dos vectors ortogonals.
  \item $\uvec{u}=\uvec{j}+6\uvec{k}$, $\uvec{v}=\uvec{i}-2\uvec{j}-\uvec{k}$. Novament no són proporcionals i, per tant, no són paral·lels. Pel que fa al producte escalar: $\uvec{u}\cdot\uvec{v}=0\cdot1+1\cdot(-2)+6\cdot(-1)=-8\neq0$. Per tant, no són ni paral·lels ni ortogonals.
\end{enumerate}
\blacksquare
