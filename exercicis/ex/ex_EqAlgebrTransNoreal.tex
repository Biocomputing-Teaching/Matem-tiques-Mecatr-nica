\Exercise Escriu una equació de cada cas:
\begin{enumerate}
    \item Escriu equacions polinòmiques amb coeficients enters que tinguin com a solucions nombres naturals, enters, racionals, irracionals (\textit{algebraics})
    \item Escriu una equació polinòmica amb coeficients enters que tingui com a solució el número $\pi$ (\textit{transcendent})
    \item Troba una equació polinòmica amb coeficients constants sense cap nombre real com a  solució
\end{enumerate}

\Answer 
\begin{enumerate}
  \item Intuitivament, una equació algebraica és aquella en la que es pot trobar la resposta usant operacions algebraiques: addició, multipliació i extracció d'arrel.
  \[2x-1=0, \; x=\frac{1}{2}\]
  o bé
  \[x^2-2=0, \; x=\pm\sqrt{2}\]
  \item Una equació transcendent, ''transcedeix'' l'àlgebra i s'han d'usar altres operacions. Per exemple, les funcions exponencials, algorítmiques o trigonomètriques. Per definició, un número real $\alpha$ és transcendent si no és algebraic. És a dir, si no existeix cap polinomi amb coeficients enters de manera que $\alpha$ en sigui una arrel. Per tant, la pregunta no té resposta, ja que no podem construir un polinomi amb solució $\pi$. La demostració \href{https://www.gaussianos.com/como-demostrar-que-%CF%80-pi-es-trascendente/}{és complicada} i la deixo com a lectura avançada.
  \item Ens passarà sempre que la solució impliqui haver de trobar l'arrel parella d'un número negatiu:
  \[x^2+1=0, \; x=\pm\sqrt{-1}\]
  que en el conjunt dels reals no té solució. Caldria resoldre-la en el conjunt dels números complexos:
  \[x^2+1=0, \; x=\pm i\]
\end{enumerate}
\blacksquare
